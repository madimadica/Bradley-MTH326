\documentclass[12pt]{article}
%\usepackage[document]{ragged2e}
\usepackage{array, amssymb, amsthm, linguex, enumerate, amsmath, physics, enumitem, xcolor, graphicx, xparse}
\let\fg\undefined %remove linguex/siunitx naming clash
\usepackage[english]{babel}
\usepackage[letterpaper,top=2cm,bottom=2cm,left=3cm,right=3cm,marginparwidth=1.75cm]{geometry}
\usepackage[colorlinks=true, allcolors=blue]{hyperref}
\usepackage[group-separator={,}]{siunitx} %\num{12345} -> "12,345"
\usepackage{fancyhdr}
\usepackage{notomath}
\usepackage[T1]{fontenc}
%Number sets
\newcommand{\R}{\mathbb{R}}
\newcommand{\C}{\mathbb{C}}
\newcommand{\N}{\mathbb{N}}
\newcommand{\F}{\mathbb{F}}
\renewcommand{\Re}{\operatorname{Re}}
\renewcommand{\Im}{\operatorname{Im}}
\renewcommand{\L}[1]{\mathcal{L}\left({#1}\right)} %Linear Map

\newcommand{\pmp}{\,\pm\,} %add small extra space to \pm

\NewDocumentCommand{\ceil}{ s m }{% ceiling brackets
    \IfBooleanTF{#1}%
    {\lceil #2 \rceil}% starred: no-autosizing
    {\left\lceil #2 \right\rceil}% unstarred: autosizing
}

\NewDocumentCommand{\ceiling}{ s m }{% ceiling brackets
    \IfBooleanTF{#1}%
    {\lceil #2 \rceil}% starred: no-autosizing
    {\left\lceil #2 \right\rceil}% unstarred: autosizing
}

\NewDocumentCommand{\floor}{ s m }{% floor brackets
    \IfBooleanTF{#1}%
    {\lfloor #2 \rfloor}% starred: no-autosizing
    {\left\lfloor #2 \right\rfloor}% unstarred: autosizing
}

\NewDocumentCommand{\pars}{ s m }{% parenthesis
    \IfBooleanTF{#1}%
    {( #2 ) }% starred: no-autosizing
    {\left( #2 \right) }% unstarred: autosizing
}

\NewDocumentCommand{\inner}{ s m }{% inner product
    \IfBooleanTF{#1}%
    {\langle #2 \rangle}% starred: no-autosizing
    {\left\langle #2 \right\rangle}% unstarred: autosizing
}

\NewDocumentCommand{\brac}{ s m }{% brackets
    \IfBooleanTF{#1}%
    {[#2] }% starred: no-autosizing
    {\left[ #2 \right] }% unstarred: autosizing
}

%default latex bracket size naming
\newcommand{\biggbrac}[1]{\bigg[ {#1} \bigg] }
\newcommand{\bigbrac}[1]{\big[ {#1} \big] }
\newcommand{\Bigbrac}[1]{\Big[ {#1} \Big] }


\RenewDocumentCommand{\over}{ s m }{% fraction 1/arg
    \IfBooleanTF{#1}%
    {\dfrac{1}{#2}}% starred: dfrac
    {\frac{1}{#2}}% unstarred: normal frac
}

\NewDocumentCommand{\pover}{ s m }{% parenthesis around fraction (1/arg)
    \IfBooleanTF{#1}%
    {\left(\dfrac{1}{#2}\right)}% starred: dfrac
    {\left(\frac{1}{#2}\right)}% unstarred: normal frac
}

\NewDocumentCommand{\pfrac}{ s m m}{% parenthesis around fraction (arg1/arg2)
    \IfBooleanTF{#1}%
    {\left( \dfrac{{#2}}{{#3}} \right)}% starred: dfrac
    {\left( \frac{{#2}}{{#3}} \right)}% unstarred: normal frac
}


\newcommand{\Xbar}{\bar{X}}
\newcommand{\Ybar}{\bar{Y}}
\newcommand{\xbar}{\bar{x}}
\newcommand{\ybar}{\bar{y}}


\newcommand{\limn}{\lim_{n\to\infty}}

\newcommand{\gammaDist}[2]{\operatorname{Gamma} \left( {#1},{#2} \right)} %gamma distribution
\NewDocumentCommand{\normalDist}{s g g}{ %normal distibution
    \IfBooleanTF{#1} { % starred, no autosizing parenthesis
      \IfNoValueTF{#2}{ 
          N (\mu,\, \sigma^2 ) %\normalDist* "default" normal distribution N(\mu, \sigma^2)
        } {
            \IfNoValueTF{#3}{N (#2)}{} %\normalDist{arg} --> N(arg)
        }
      \IfNoValueTF{#3}{}{N ( #2, #3 )}  %\normalDist*{arg1}{arg2} --> N(arg1,arg2)
    }  % else (unstarred) autosize parenthesis
    {
        \IfNoValueTF{#2}{
            N \left(\mu,\, \sigma^2 \right) %\normalDist "default" normal distribution N(\mu, \sigma^2)
        } {
            \IfNoValueTF{#3}{N \left(#2\right)}{} %\normalDist{arg} --> N(arg)
        }
        \IfNoValueTF{#3}{}{N \left( #2, #3 \right)} %\normalDist{arg1}{arg2} --> N(arg1,arg2)
    }
}



%colors
\definecolor{ggreen}{RGB}{0, 127, 0}
\definecolor{dgray}{RGB}{63,63,63}
\definecolor{neonorange}{RGB}{255,47,0}
\definecolor{mygray}{rgb}{0.5,0.5,0.5}
\definecolor{eblue}{RGB}{0,74,127}
\newcommand{\red}[1]{\color{red}{#1}\color{black}}
\newcommand{\green}[1]{\color{ggreen}{#1}\color{black}}
\newcommand{\blue}[1]{\color{blue}{#1}\color{black}}
\newcommand{\setRed}{\color{red}}
\newcommand{\setBlack}{\color{black}}
\newcommand{\setBlue}{\color{blue}}
\newcommand{\setGreen}{\color{ggreen}}



\newcommand{\thru}[1]{{#1}_1, \dots, {#1}_n}
\newcommand{\sumThru}[1]{{#1}_1 + \cdots + {#1}_n}
\newcommand{\yn}{Y_1, \dots, Y_n} % Y_1, ..., Y_n
\newcommand{\xn}{X_1, \dots, X_n} % Y_1, ..., Y_n

%hats and tildes
\newcommand{\that}{\widehat{\theta}} % theta hat
\newcommand{\phat}{\widehat{p}} % p hat
\newcommand{\qhat}{\widehat{q}} % p hat
\newcommand{\psihat}{\widehat{\psi}} % psi hat
\newcommand{\Psihat}{\widehat{\Psi}} % Psi hat
\newcommand{\ptilde}{\widetilde{p}} % psi tilde
\newcommand{\Psitil}{\widetilde{\Psi}} % Psi tilde
\newcommand{\betah}{\widehat{\beta}} % beta hat

%2x2 matrix shortcuts
\newcommand{\detx}[4]{\begin{vmatrix}{#1} & {#2}\\{#3}&{#4}\end{vmatrix}} % 2x2 determinant
\newcommand{\bmat}[4]{\begin{bmatrix}{#1} & {#2}\\{#3}&{#4}\end{bmatrix}} % 2x2 matrix brackets
\renewcommand{\pmat}[4]{\begin{pmatrix}{#1} & {#2}\\{#3}&{#4}\end{pmatrix}} % 2x2 matrix parenthesis

%remove any enumerate/itemize indent temporarily
\makeatletter   %% <- make @ usable in macro names
\newcommand*\notab[1]{%
  \begingroup   %% <- limit scope of the following changes
    \par        %% <- start a new paragraph
    \@totalleftmargin=0pt \linewidth=\columnwidth
    %% ^^ let other commands know that the margins have been reset
    \parshape 0
    %% ^^ reset the margins
    #1\par      %% <- insert #1 and end this paragraph
  \endgroup
}
\makeatother    %% <- revert @


\newcommand{\dimrange}[1]{\operatorname{dim}\operatorname{range}{#1}} % dimrange
\newcommand{\dimnull}[1]{\operatorname{dim}\operatorname{null}{#1}} % dimnull
\newcommand{\range}[1]{\operatorname{range}{#1}} %range
\newcommand{\nullspace}{\operatorname{null}} %null

% polynomial notation
\NewDocumentCommand{\poly}{ s g g }{%
    \IfBooleanTF{#1} {
        \IfNoValueTF{#2} {
            \mathcal{P}(\mathbb{R})
        } {
            \mathcal{P}_{#2}(\mathbb{R})
        }
    } {
        \IfNoValueTF{#3} {
            {\mathcal{P}(#2)}
        } { %else
            {\mathcal{P}_{#2}(#3)}
        }
    }
}

\NewDocumentCommand{\bias}{ s m }{% bias(arg)
    \IfBooleanTF{#1}%
    {\operatorname{bias}(#2)}% starred: no autosizing
    {\operatorname{bias}\left(#2\right)}% unstarred: autosizing
}

\NewDocumentCommand{\MSE}{ s m }{% MSE(arg)
    \IfBooleanTF{#1}%
    {\operatorname{MSE}(#2)}% starred: no autosizing
    {\operatorname{MSE}\left(#2\right)}% unstarred: autosizing
}

\NewDocumentCommand{\Var}{ s m }{% variance with parenthesis V(arg)
    \IfBooleanTF{#1}%
    {\operatorname{Var}(#2)}% starred: no autosizing
    {\operatorname{Var}\left(#2\right)}% unstarred: autosizing
}

\NewDocumentCommand{\Varb}{ s m }{% variance with brackets V[arg]
    \IfBooleanTF{#1}%
    {\operatorname{Var}[\,#2\,]}% starred: no autosizing
    {\operatorname{Var}\left[\,#2\,\right]}% unstarred: has autosizing
}

\NewDocumentCommand{\Vb}{ s m }{% another renaming of variance with brackets V[arg]
    \IfBooleanTF{#1}%
    {\operatorname{Var}[\,#2\,]}% starred: no autosizing
    {\operatorname{Var}\left[\,#2\,\right]}% unstarred: has autosizing
}

\NewDocumentCommand{\E}{ s m }{% expectation with parenthesis E(arg)
    \IfBooleanTF{#1}%
    {\operatorname{E}(#2)}% starred: no autosizing
    {\operatorname{E}\left(#2\right)}% unstarred: has autosizing
}

\NewDocumentCommand{\Eb}{ s m }{% expectation with brackets E[arg]
    \IfBooleanTF{#1}%
    {\operatorname{E}[#2]}% starred: no autosizing
    {\operatorname{E}\left[#2\right]}% unstarred: has autosizing
}

\RenewDocumentCommand{\P}{ s m }{% probability with parenthesis Pr(arg)
    \IfBooleanTF{#1}%
    {\Pr (#2) }% starred: no autosizing
    {\Pr \left( #2 \right) }% unstarred: has autosizing
}

\NewDocumentCommand{\prob}{ s m }{% probability with parenthesis Pr(arg)
    \IfBooleanTF{#1}%
    {\Pr (#2) }% starred: no autosizing
    {\Pr \left( #2 \right) }% unstarred: has autosizing
}

\NewDocumentCommand{\eff}{ s m }{% efficiency with parenthesis eff(arg)
    \IfBooleanTF{#1}%
    {\operatorname{eff}(#2)}% starred: no autosizing
    {\operatorname{eff}\left(#2\right)}% unstarred: has autosizing
}

%vertical vector of up to 8 elements
\NewDocumentCommand\vvec{s m g g g g g g g}{%
    \IfBooleanTF{#1} {
        \begin{bmatrix}% if starred use brackets
            \IfNoValueTF{#2}{}{#2}
            \IfNoValueTF{#3}{}{\\#3}
            \IfNoValueTF{#4}{}{\\#4}
            \IfNoValueTF{#5}{}{\\#5}
            \IfNoValueTF{#6}{}{\\#6}
            \IfNoValueTF{#7}{}{\\#7}
            \IfNoValueTF{#8}{}{\\#8}
        \end{bmatrix}
    }  % else (unstarred) use parethesis
    {
        \begin{pmatrix}%
            \IfNoValueTF{#2}{}{#2}
            \IfNoValueTF{#3}{}{\\#3}
            \IfNoValueTF{#4}{}{\\#4}
            \IfNoValueTF{#5}{}{\\#5}
            \IfNoValueTF{#6}{}{\\#6}
            \IfNoValueTF{#7}{}{\\#7}
            \IfNoValueTF{#8}{}{\\#8}
        \end{pmatrix}
    }
}
\def\Cov{\operatorname{Cov}} %Covariance
\def\df{\text{df}} %degrees of freedom

\NewDocumentCommand{\example}{ s g }{% Example header
    \IfBooleanTF{#1}%
    {\vspace{0.1in}}% starred: 0.1in
    {\vspace{0.2in}}% unstarred: 0.2in
    \IfNoValueTF{#2} {\noindent\textbf{\color{eblue} Example: }}{\noindent\textbf{\color{eblue} Example (#2): }}
}
\NewDocumentCommand{\disc}{ s }{% Discussion header
    \IfBooleanTF{#1}%
    {\vspace{0.1in}\noindent\textbf{Discussion: } }% starred: 0.1in
    {\vspace{0.2in}\noindent\textbf{Discussion: } }% unstarred: 0.2in
}
\NewDocumentCommand{\defn}{ s }{% Definition header
    \IfBooleanTF{#1}%
    {\vspace{0.1in}\noindent\textbf{\color{neonorange} Definition: } }% starred: 0.1in
    {\vspace{0.2in}\noindent\textbf{\color{neonorange} Definition: } }% unstarred: 0.2in
}
\NewDocumentCommand{\reason}{ s }{% Reason header
    \IfBooleanTF{#1}%
    {\vspace{0.1in}\noindent\textbf{Reason:} }% starred: 0.1in
    {\vspace{0.2in}\noindent\textbf{Reason:} }% unstarred: 0.2in
}
\NewDocumentCommand{\recall}{ s }{% Recall header
    \IfBooleanTF{#1}%
    {\vspace{0.1in}\noindent\textit{Recall:} }% starred: 0.1in
    {\vspace{0.2in}\noindent\textit{Recall:} }% unstarred: 0.2in
}
\NewDocumentCommand{\remark}{ s }{% Remark header
    \IfBooleanTF{#1}%
    {\vspace{0.1in}\noindent\textit{Remark:} }% starred: 0.1in
    {\vspace{0.2in}\noindent\textit{Remark:} }% unstarred: 0.2in
}

\newcommand{\proj}[2]{\operatorname{proj}_{{#1}}{#2}} %projection
\newcommand{\wideand}{\qquad \text{and} \qquad}

\newcommand{\bu}[1]{\textbf{\underline{{#1}}} } %bold underline
\newcommand{\boldit}[1]{\textbf{\textit{{#1}}} } %bold italix

% put actual quotation marks "around something"
\newcommand{\say}[1]{\textquotedblleft{#1}\textquotedblright}

% max{arg} and min{arg}
\renewcommand{\max}[1]{\operatorname{max}\left\{ #1 \right\}}
\renewcommand{\min}[1]{\operatorname{min}\left\{ #1 \right\}}

%circle around argument
\newcommand{\cir}[1]{\textcircled{\raisebox{-1pt}{#1}}}

% Updates the current chapter number.
% 0 based indices
\newcommand{\setChapter}[1]{
    \setcounter{chapter}{#1}
    \setcounter{section}{0}
}



% Update global section number, O based indices
\newcommand{\setSection}[1]{\setcounter{section}{#1}}



%Set table of contents depth
\setcounter{tocdepth}{3}


%Create a new vspace line no indent
\newcommand{\nl}{\vspace{0.1in}\noindent}
\newcommand{\nnl}{\vspace{0.2in}\noindent}
\newcommand{\nnnl}{\vspace{0.3in}\noindent}

%Remove hyphens
\tolerance=1
\emergencystretch=\maxdimen
\hyphenpenalty=10000
\hbadness=10000

\textwidth=7.02in
\hoffset=-.45in



\setcounter{MaxMatrixCols}{20}
\begin{document}

\pagestyle{fancy}
\fancyhf{}
\fancyhead[RO]{Matthew Wilder} %header top right
\fancyhead[LO]{MTH 326 - Homework \#8} %header top left
\fancyfoot[CO]{Page \thepage} %page center bottom

\noindent MTH 326 - Spring 2022
\\Assignment \#8
\\Due: Wednesday, April 13 2022 (11:59pm)

\begin{enumerate}
    \item Suppose the following represents a random sample of points $(x,y)$:
\notab{\begin{center}
    \begin{tabular}{|c|c|c|c|c|c|} 
         \hline
         $x$ & -2 & -1 & 0 & 1 & 2\\
         \hline
         $y$ & 3 & 2 & 1 & 1 & 0.5\\
         \hline
    \end{tabular}
\end{center}}

\begin{enumerate}[label={(\alph*)}]
    \item Find the 90\% confidence interval for $\E{Y}$ when $x^* = 0$ and again when $x^* = 2$.

\nnl \textbf{Solution: } For a 90\% CI then $\alpha = 1 - 0.9 = 0.1$ and a two sided is $\alpha/2 = 0.05$. There are $n-2$ degrees of freedom, hence $\df = 5-2 = 3$. Therefore $t_{\alpha/2}(\df) = t_{0.05}(3) = 2.353.$ We want to use the following formula,
$$\pars{\widehat{\beta_0} + \widehat{\beta_1} x^*} \, \pm \, t_{\alpha/2}(\text{df})S \displaystyle \sqrt{\over{n} + \dfrac{\pars{x^* - \bar x}^2}{S_{xx}}}$$
so we need to compute the remaining unknows. It is trivial that $\xbar = 0$ and $\ybar = 1.5$. Then
$$S_{xx} = \sum (x_i-\xbar)^2 = \sum x_i^2 = 4 + 1 + 0 + 1 + 4 = 10$$
$$S_{yy} = \sum (y_i-\ybar)^2 = \sum (y_i-1.5)^2 =  (1.5)^2 + (0.5)^2 + 2(-0.5)^2 + (-1)^2 = 4$$
and 
$$S_{xy} = \sum (x_i-\xbar)(y_i-\ybar) = \sum x_i(y_i-1.5) = -2(1.5) - (0.5) + (-0.5) + 2(-1) = -6.$$
Then 
$$\betah_1 = \frac{S_{xy}}{S_{xx}} = \frac{-6}{10} = -0.6$$
and
$$\betah_0 = \ybar - \betah_1 \xbar = 1.5.$$
Then
$$S = \sqrt{\dfrac{\operatorname{SSE}}{n-2}} =  \sqrt{ \dfrac{S_{yy} - \betah_1 S_{xy}} {n-2}} = \sqrt{ \dfrac{4- (-0.6 \cdot -6)}{3} } = \sqrt{\frac{0.4}{3}} \approx  0.365148.$$
Therefore our inverval is
$$(1.5 - 0.6x^*) \pm 2.353\cdot  0.365148 \sqrt{\over{5} + \frac{(x^*)^2}{10}}$$
so for $x^* = 0$ then 
$$1.5 \pm 2.353\cdot  0.365148 \sqrt{\over{5}} \equiv (1.116, 1.884)$$
and for $x^* = 2$ then 
$$(1.5 - 0.6(2) ) \pm 2.353\cdot  0.365148 \sqrt{\over{5} + \frac{(2)^2}{10}} \equiv (-0.366, 0.966).$$
    \item Find the 90\% confidence interval for $Y^*$ when $x^* = 0$ and again when $x^* = 2$.

\nnl \textbf{Solution: } We adjust the interval to 
$$(1.5 - 0.6x^*) \pm 2.353\cdot  0.365148 \sqrt{{\setRed 1} + \over{5} + \frac{(x^*)^2}{10}}$$
so for $x^* = 0$ then 
$$1.5 \pm 2.353\cdot  0.365148 \sqrt{1 + \over{5}} \equiv (0.5588, 2.4412)$$
and for $x^* = 2$ then 
$$(1.5 - 0.6(2) ) \pm 2.353\cdot  0.365148 \sqrt{1 + \over{5} + \frac{(2)^2}{10}} \equiv (-0.7868, 1.3868).$$\vspace{1in}
    \item Are the intervals for $\E{Y}$ or the intervals for $Y^*$ wider? How can this be explained?

\nnl \textbf{Solution: } The intervals for $Y^*$ are wider than $\E{Y}$ because the $(\betah_0 + \betah_1x^*)$ is held constant as the center of the interval, then the scalar $t_{\alpha/2}(\df)S$ is a fixed constant. Lastly, the scalar value under the squareroot is positive since $\frac{1}{n} > 0$, $(x^*-\xbar)^2 > 0 \, \forall \, x \in \R$, and $S_{xx} > 0$ since its also squares.  Adding positive values together is stilll positive. Hence adding another positive number (+1) to this value results in a larger positive value. Therefore the squareroot of the new positive value is also larger. Hence the scalar around the center of the interval is larger (\say{wider}).

\notab{
$$\overbrace{\pars{\widehat{\beta_0} + \widehat{\beta_1} x^*} \, \pm \, t_{\alpha/2}(\text{df})S \displaystyle \sqrt{\over{n} + \dfrac{\pars{x^* - \bar x}^2}{S_{xx}}}}^{\textstyle \E{Y}} < \overbrace{\pars{\widehat{\beta_0} + \widehat{\beta_1} x^*} \, \pm \, t_{\alpha/2}(\text{df})S \displaystyle \sqrt{1 + \over{n} + \dfrac{\pars{x^* - \bar x}^2}{S_{xx}}}}^{\textstyle Y^*}$$
$$\iff t_{\alpha/2}(\text{df})S \sqrt{\over{n} + \dfrac{\pars{x^* - \bar x}^2}{S_{xx}}} < t_{\alpha/2}(\text{df})S \sqrt{1 + \over{n} + \dfrac{\pars{x^* - \bar x}^2}{S_{xx}}}$$
$$\iff \sqrt{\underbrace{\over{n} + \dfrac{\pars{x^* - \bar x}^2}{S_{xx}}}_{> 0}} < \sqrt{1 + \underbrace{\over{n} + \dfrac{\pars{x^* - \bar x}^2}{S_{xx}}}_{>0}}$$}
\end{enumerate}
    
    \newpage
    \notab{The following table contains dietary data (calories and the content of fat, sodium, carbohydrate, and protein) in some standard hamburgers that can be found at local fast food
    restaurants.}
    \notab{\begin{center}
        \begin{tabular}{|l|c|c|c|c|c|} 
             \hline
             & cal & fat (g) & sodium (mg) & carbs (g) & protein (g)\\
             \hline
             BK Jr. & 310 & 18 & 390 & 27 & 13\\
             Wendy's Jr. & 250 & 11 & 420 & 25 & 13\\
             McDonald's & 250 & 9 & 480 & 31 & 12\\
             Culvers & 390 & 17 & 480 & 38 & 20\\
             Steak-n-Shake & 320 & 14 & 830 & 32 & 15\\
             Sonic Jr. & 330 & 16 & 610 & 32 & 15\\
             \hline
        \end{tabular}
    \end{center}}

    \item Data is collected from three populations ($A$, $B$, $C$) which have normal distributions with a common variance. The data is as follows:
\notab{
    \begin{center}
    \begin{tabular}{|c|c c c|} 
        \hline
        A & 1 & 3 & 2\\
        B & 6 & 4 & \\
        C & 6 & 8 & 4\\
        \hline
   \end{tabular}\end{center}
}
\begin{enumerate}[label={(\alph*)}]
    \item This data set is small enough to do by hand. Determine $\widehat{\mu}_i$'s, $\widehat{\mu}_0$, $\operatorname{SSA}$ and $\operatorname{SSW}$.

\soln We have $k=3$, $n_1 = n_3 = 3$, $n_2 = 2$ and $n = 8$. Then
\begin{align*}
    \muh_1 &= \over{3} \bigbrac{1 + 3 + 2} = 2\\
    \muh_2 &= \over{2} \bigbrac{6 + 4} = 5\\
    \muh_3 &= \over{3} \bigbrac{6+8+4} = 6\\
    \muh_0 &= \over{n} \sum_{i=1}^{k} n_i \muh_i = \over{8} \bigbrac{3(2)+2(5)+3(6)} = 4.25\\
    \operatorname{SSA} &= \sum_{i=1}^k n_i (\muh_i - \muh_0)^2 = 3 (2 - 4.25)^2 + 2 (5 - 4.25)^2 + 3 (6 - 4.25)^2 = 25.5\\
    \operatorname{SSW} &= \sum_{i=1}^k \sum_{j=1}^{n_i} \pars{Y_{i\,j} - \muh_i}^2\\
    &= \underbrace {
        (1-2)^2 + (3-2)^2 + (2-2)^2
    }_{i=1}
    + \underbrace{
        (6-5)^2 + (4-5)^2
    }_{i=2} + \underbrace{
        (6-6)^2 + (8-6)^2 + (4-6)^2
    }_{i=3}
    = 12
\end{align*}
    \item Fill out the ANOVA table for this experiment.

\soln* 
\notab{
    \begin{center}
    \begin{tabular}{|l|c c c c c|} 
        \hline
        Source & SS & df & Mean Squares & $F_{\text{obs}}$ & $p$-value\\
        \hline
        among & 25.5 & 2 & 12.75 & 5.3125 & 0.057926\\
        within & 12 & 5 & 2.4 & &\\
        total & 37.5 & 7 & & &\\
        \hline
   \end{tabular}
\end{center}
}\newpage
    \item Is there sufficient evidence at level $\alpha = 0.10$ that at least one of the means is different from the others? Briefly explain.

\soln* Since $0.058 = p < \alpha = 0.1$ then we reject the null hypothesis. Hence, there \textit{is} evidence that at least one of the means are different. \vspace{0.3in}
    \item Give bounds for the $p$-value in this experiment.

\soln*$p \in (0.05, 0.1)$ by appendix table 7 since $F(2,5)_{.1} = 3.78 <F < F(2,5)_{.05} = 5.79$
\end{enumerate}
    \newpage
    \item Students at a large high-school preparing for the ACT exam have the option of attending
an ACT tutorial session. Some students chose to do this and some do not. School
administrators want to determine if the tutorial helps. The ACT scores of students at
this school are summarized below:
\notab{\begin{center}
    \begin{tabular}{|r l l l|} 
         \hline
         attended tutorial: & $\ybar = 26$ & $S = 3.5$ & $n = 50$ \\
         did not attend: & $\ybar = 25$ & $S = 3$ & $n = 80$\\ 
         \hline
    \end{tabular}
\end{center}}
\begin{enumerate}[label={(\alph*)}]
    \item Determine $\widehat{\mu}_0$ and then compute $\operatorname{SSA}$ (or $\operatorname{SST}$, using the book's notation) and $\operatorname{SSW}$ (or $\operatorname{SSE}$) in the book.

\soln* 
\begin{align*}
    \muh_0 &= \over{n} \sum_{i=1}^{k} n_i \muh_i = \over{130} \bigbrac{50(26)+80(25)} = 25.3846154\\
    \operatorname{SSA} &= \sum_{i=1}^k n_i (\muh_i - \muh_0)^2 = 50 (26 - 25.3846154)^2 + 80 (25 - 25.3846154)^2 = 30.7692308\\
    \operatorname{SSW} &= (50 - 1)(3.5^2) + (80-1)(3^2) = 1311.25\\
\end{align*}\vspace{.2in}
    \item Fill out the ANOVA table and test the administrators' conjecture at the $\alpha =0.05$ significance level. Fully justify your conclusion.

\soln* 
\notab{
    \begin{center}
    \begin{tabular}{|l|c c c c c|} 
        \hline
        Source & SS & df & Mean Squares & $F_{\text{obs}}$ & $p$-value\\
        \hline
        among & 30.769 & 1 & 30.769 & 3.00361187 & 0.085488\\
        within & 1311.25 & 128 & 10.244 & &\\
        total & 1342.02 & 129 & & &\\
        \hline
   \end{tabular}
\end{center}
}
Let $H_0 : \mu_1 = \mu_2$ and $H_a : \mu_1 \neq \mu_2$. Then for $\alpha = 0.05$, computing $F_{0.05}(1, 128) = 3.9151$ by WebAssign technology. Since $F_{\text{obs}} < F = 3.9151$ then we accept $H_0$. There is not enough evidence to indicate a difference in ACT performance.\vspace{.2in}
    \item Determine a 95\% confidence interval for the difference in means.

\soln* Using Chapter 9 techniques, $\displaystyle \text{CI} \equiv (\muh_1 - \muh_2) \pm t_{\alpha/2}(\df)\sqrt{\frac{S_1}{n_1} + \frac{S_2}{n_2}}$. Hence
\begin{align*}
    \text{C.I.} &= (26 - 25) \pm t_{0.025}(129) \sqrt{ \frac{3.5^2}{50} + \frac{3^2}{80}}\\
    &= 1 \pm 1.960 \cdot 0.59791\\
    &\equiv (-0.1719, \; 2.1719) \text{ points}
\end{align*}
Using $\operatorname{MSW} \equiv S^2$ yields (as it should) about the same decimals. \vspace{0.2in}
    \item Explain how your answer in (b) supports your conclusion in (a).

\soln* Interpretting this as how (b) supports (c): Since we accept the null hypothesis in (b) it would make logical sense that $0 \in \text{CI}$ from (c).
\end{enumerate}
    \end{enumerate}
\end{document} 