\documentclass[12pt]{article}
\usepackage[english]{babel}
\usepackage{array}
\usepackage{setspace}
\usepackage{graphicx}
\usepackage{sistyle} %\num{100000} for commas
\SIthousandsep{,}
\usepackage{fancyhdr}
\usepackage{listings} % For code listings, may break stuff
\usepackage{xcolor, diagbox, empheq, makecell, tcolorbox}
\usepackage[autostyle]{csquotes}
\usepackage{amssymb, amsthm, linguex, enumitem, amsmath}
\usepackage{tcolorbox} %dont know what this does
\usepackage[colorlinks=true, allcolors=blue]{hyperref}
\usepackage{lipsum}

\makeatletter   %% <- make @ usable in macro names
\newcommand*\notab[1]{%
  \begingroup   %% <- limit scope of the following changes
    \par        %% <- start a new paragraph
    \@totalleftmargin=0pt \linewidth=\columnwidth
    %% ^^ let other commands know that the margins have been reset
    \parshape 0
    %% ^^ reset the margins
    #1\par      %% <- insert #1 and end this paragraph
  \endgroup
}
\makeatother    %% <- revert @

\pagestyle{empty}

\textwidth 6.5in
\hoffset=-.65in
\textheight=9.5in
\voffset=-1.in

%Sets
\newcommand{\R}{\mathbb{R}}
\newcommand{\C}{\mathbb{C}}
\newcommand{\N}{\mathbb{N}}
\newcommand{\F}{\mathbb{F}}
\newcommand{\Z}{\mathbb{Z}}
\newcommand{\Cb}{\mathbf{C}}
\newcommand{\Fb}{\mathbf{F}}
\newcommand{\Rb}{\mathbf{R}}

%Misc

\newcommand{\pars}[1]{\left( {#1} \right) } %auto size parenthesis 
\newcommand{\brac}[1]{\left[ {#1} \right] } %auto size brackets around arg
\newcommand{\bbrac}[1]{\bigg[ {#1} \bigg] } %auto size brackets around arg
\newcommand{\set}[1]{\left\{{#1}\right\}} %auto size curly braces around arg
\newcommand{\vbrac}[1]{\left\langle{#1}\right\rangle} %vector angle brackets
\newcommand{\inner}[1]{\left\langle{#1}\right\rangle} %vector angle brackets
\newcommand{\conj}[1]{\overline{{#1}}} %conjugate bar
\newcommand{\vconj}[1]{\overline{\vbrac{{#1}}}}
\newcommand{\ceil}[1]{\left\lceil {#1} \right\rceil} %auto size ceiling around arg
\newcommand{\floor}[1]{\left\lfloor {#1} \right\rfloor} %auto size floor around arg

\newcommand{\limn}{\lim_{n\to\infty}} %limit as n approaches infinity
\newcommand{\thru}[1]{{#1}_1, \dots, {#1}_n}
\newcommand{\sumthru}[1]{{#1}_1 + \cdots + {#1}_n}
\renewcommand{\over}[1]{\frac{1}{{#1}}}
\newcommand{\pfrac}[2]{\left( \frac{{#1}}{{#2}} \right) } %auto size parenthesis over fraction 
\newcommand{\pover}[1]{\left( \frac{1}{{#1}} \right) } %auto size parenthesis over fraction

%Boolean Algebra
\newcommand{\OR}{\,\lor\,}
\newcommand{\AND}{\,\land\,}

%Probability and Statistics
\newcommand{\xbar}{\bar{X}}
\newcommand{\ybar}{\bar{Y}}
\newcommand{\yn}{Y_1, \dots, Y_n}
\newcommand{\yx}{X_1, \dots, X_n}
\newcommand{\normDist}{N\left(\mu, \sigma^2\right)} %default normal distribution
\newcommand{\gammaDist}[2]{\operatorname{Gamma} \left( {#1},{#2} \right)}
\newcommand{\norm}[1]{\left\| {#1} \right\|}
\newcommand{\abs}[1]{\left| {#1} \right|}
\newcommand{\prob}[1]{P \left( {#1} \right) }
\newcommand{\E}[1]{\operatorname{E} \left( {#1} \right) }
\newcommand{\Eb}[1]{\operatorname{E} \brac{ \,{#1}\, }} %E bracket
\newcommand{\Es}[1]{\operatorname{E}[ \,{#1}\, ]} %E bracket
\newcommand*{\V}[1]{\operatorname{V} \left( {#1} \right) }
\newcommand{\Vb}[1]{\operatorname{V} \brac{ \,{#1}\, } }
\newcommand{\Vs}[1]{\operatorname{V} [\,{#1}\, ] }
\newcommand{\that}{\widehat{\theta}} %theta hat
\newcommand{\phat}{\hat{p}}
\newcommand{\psihat}{\hat{\psi}}
\newcommand{\Psihat}{\hat{\Psi}}
\newcommand{\eff}[2]{\operatorname{eff}({#1},\,{#2})}


\newcommand{\dimrange}[1]{\operatorname{dim}\operatorname{range}{#1}} % dimrange
\newcommand{\dimnull}[1]{\operatorname{dim}\operatorname{null}{#1}} % dimnull

\newcommand{\vdub}[2]{\begin{pmatrix}{#1}\\{#2}\end{pmatrix}}

%Linear Algebra
\newcommand{\poly}[1]{\mathcal{P}_{#1}(\mathbf{R})} %polynomial up to degree (arg)
\newcommand{\pf}{\mathcal{P}(\mathbf{F})} %set of all polynomials
\renewcommand{\L}[1]{\mathcal{L}\left({#1}\right)}%Set of linear maps
\newcommand{\vdouble}[2]{\begin{pmatrix}{#1}\\{#2}\end{pmatrix}} % 2 high vertical
\newcommand{\vtriple}[3]{\begin{pmatrix}{#1}\\{#2}\\{#3}\end{pmatrix}} %vertical vector parenthesis, 3 args
\newcommand{\vquad}[4]{\begin{pmatrix}{#1}\\{#2}\\{#3}\\{#4}\end{pmatrix}} %vertical vector parenthesis, 4 args
\newcommand{\vquin}[5]{\begin{pmatrix}{#1}\\{#2}\\{#3}\\{#4}\\{#5}\end{pmatrix}} %vertical vector parenthesis, 5 args
\newcommand{\vvector}[3]{\begin{bmatrix}{#1}\\{#2}\\{#3}\end{bmatrix}} %vertical vector braces, 3 args
\newcommand{\dimn}[1]{\operatorname{dim}\,{#1}}
\newcommand{\rank}[1]{\operatorname{dim}\operatorname{range}{#1}}
\newcommand{\nullity}[1]{\operatorname{dim}\operatorname{null}{#1}}
\newcommand{\range}[1]{\operatorname{range}{#1}}
\newcommand{\NULL}[1]{\operatorname{null}{#1}}
\renewcommand{\null}{\operatorname{null}}
\newcommand{\mat}[4]{\begin{bmatrix}{#1} & {#2}\\{#3}&{#4}\end{bmatrix}} % 2x2 matrix
\newcommand{\Mat}[9]{\begin{bmatrix}{#1} & {#2} & {#3}\\{#4}&{#5}&{#6}\\{#7}&{#8}&{#9}\end{bmatrix}} % 3x3 matrix
\newcommand{\detx}[4]{\begin{vmatrix}{#1} & {#2}\\{#3}&{#4}\end{vmatrix}} % 2x2 determinant
\newcommand{\nf}{\infty}
%colors
\definecolor{ggreen}{RGB}{0, 127, 0}
\definecolor{dgray}{RGB}{63,63,63}
\definecolor{neonorange}{RGB}{255,47,0}
\definecolor{mygray}{rgb}{0.5,0.5,0.5}
\newcommand{\red}[1]{\color{red}{#1}\color{black}}
\newcommand{\grn}[1]{\color{ggreen}{#1}\color{black}}
\newcommand{\blu}[1]{\color{blue}{#1}\color{black}}
\newcommand{\redx}[1]{\color{red}\not{#1}\color{black}}

\newcommand{\say}[1]{\textquotedblleft{#1}\textquotedblright} %quote the "argument"
\newcommand*\widefbox[1]{\fbox{\hspace{2em}#1\hspace{2em}}}
\newtcolorbox{mybox}[1][]{colback=white, sharp corners, #1}

%Line break spacings
\newcommand{\nl}{\vspace{0.1in}\noindent}
\newcommand{\nnl}{\vspace{0.2in}\noindent}
\newcommand{\nnnl}{\vspace{0.3in}\noindent}

% Code snippets
\newcommand*{\code}{\fontfamily{qcr}\selectfont}
\lstset{
    backgroundcolor=\color{white},
    basicstyle=\footnotesize,
    breakatwhitespace=false,         % sets if automatic breaks should only happen at whitespace
    breaklines=true,                 % sets automatic line breaking
    captionpos=b,                    % sets the caption-position to bottom
    commentstyle=\color{dgray},    % comment style
    deletekeywords={...},            % if you want to delete keywords from the given language
    escapeinside={(*@}{@*)},          % if you want to add LaTeX within your code
    extendedchars=true,              % lets you use non-ASCII characters; for 8-bits encodings only, does not work with UTF-8
    firstnumber=1,                % start line enumeration with line 1
    frame=single,	                   % adds a frame around the code
    keepspaces=true,                 % keeps spaces in text, useful for keeping indentation of code (possibly needs columns=flexible)
    keywordstyle=\color{neonorange},       % keyword style
    language=C++,                 % the language of the code
    morekeywords={*,...},            % if you want to add more keywords to the set
    numbers=left,                    % where to put the line-numbers; possible values are (none, left, right)
    numbersep=5pt,                   % how far the line-numbers are from the code
    numberstyle=\tiny\color{mygray}, % the style that is used for the line-numbers
    rulecolor=\color{black},         % if not set, the frame-color may be changed on line-breaks within not-black text (e.g. comments (green here))
    showspaces=false,                % show spaces everywhere adding particular underscores; it overrides 'showstringspaces'
    showstringspaces=false,          % underline spaces within strings only
    showtabs=false,                  % show tabs within strings adding particular underscores
    stringstyle=\color{purple},     % string literal style
    tabsize=4,	                   % sets default tabsize to 4 spaces
}

\lstdefinestyle{cpp}{language=C++,
    morekeywords={cout, cin, Comparable, T},numbers=none
}
%Examples:
%{\code while}
%
%{\code \begin{lstlisting}[language=C++]
%sum1 = 0;
%for (i = 1; i <= n; i *= 2)
%    for (j = 1; j <= n; j++)
%        sum1++;
%\end{lstlisting}}







\begin{document}

\pagestyle{fancy}
\fancyhf{}
\fancyhead[RO]{Matthew Wilder} %header top right
\fancyhead[LO]{MTH 326 - Exam \#1 Corrections} %header top left
\fancyfoot[CO]{Page \thepage} %page center bottom

\noindent MTH 326 - Spring 2022
\\Exam 1 Corrections
\\Due: Monday March 21, 2022 (11:59pm)

\begin{enumerate}
    \item (5 points) In a study to compare the perceived effects of two pain relievers, 200 patients
were given medicine $A$, of whom 90\% found relief, and 300 patients were given medicine
$B$ with 80\% experiencing relief. Find a 95\% confidence interval for the difference in
population proportions experiencing relief between $A$ and $B$.
\begin{mybox}
    We want $\mu_A - \mu_B$ on a large sample $n > 30$ so we will need to use a $\mathcal{Z}\text{-test}$. For a 95\% confidence interval ($\alpha = 0.05$) we use a $\mathcal{Z}\text{-score}$ of $\mathcal{Z}_{0.025} = 1.960$.
    
    \nl We have $\displaystyle \text{CI} \equiv \mu_1 - \mu_2 \pm \mathcal{Z}_{\alpha/2} \sqrt{\dfrac{p_1(1-p_1)}{n_1} + \dfrac{p_2(1-p_2)}{n_2}}$. Substituting in our given information with $A \equiv 1$ and $B \equiv 2$,
    \begin{align*}
        \text{C.I.} &= 0.9 - 0.8 \pm 1.960 \sqrt{\dfrac{0.9(1-0.9)}{200} + \dfrac{0.8(1-0.8)}{300}}\\
        &\approx (0.03853, \; 0.16146) .
    \end{align*}
\end{mybox}
    \vspace{.25in}
    %\newpage
    \item (10 points) Air trapped in amber from the Cretaceous era (75 million years ago) may
suggest that the composition of our atmosphere has changed. Nine different samples
have been obtained and the gas tested for the percentage of nitrogen in the atmosphere.
We will treat these as a random sample.
\begin{enumerate}[label=(\alph*)]
\item Given that $\xbar = 59.6\%$ and $S^2 = 39.13$, compute a 99\% confidence interval on the
nitrogen level in the ancient atmosphere. (FYI, the nitrogen level of our air is 78.1\%
today.)
\begin{mybox}
    We are given that $\alpha = 0.01$, $n=9$, and therefore d.f.$\,= 8$. Since $n < 30$ we need a $t$-test and will use the following formula: $\text{CI} = \xbar \pm t_{\alpha/2}(\text{d.f.}) \cdot \text{SE}$, where $\text{SE} = \dfrac{\sigma}{\sqrt{n}} = \sqrt{\dfrac{\sigma^2}{n}}$. We have $t_{0.005}(8) = 3.355$ and hence
    \begin{align*}
        \text{C.I.} &= 59.6\% \pm 3.355 \cdot \sqrt{\frac{39.13\%}{9}} = \pars{52.60438\%,\, 66.59561\%}.
    \end{align*}
\end{mybox}


\item Construct a 90\% confidence interval for the population variance $\sigma^2$.
\begin{mybox}
    For the 90\% \textit{lower} tail, using a modified formula $\displaystyle \pars{0, \; \dfrac{(n-1)S^2}{\chi^2_{\alpha}}}$ and with $\alpha = 0.10$, $n=9$, $\text{df}=8$, $S^2 = 39.13\%$. Then $\chi^2_{0.1}(8) \approx 13.36$ and thus, $$\text{C.I.} \equiv \pars{0, \; \dfrac{8\cdot 39.13\%}{13.36}} \approx \pars{0\%,\; 23.43113\%}.$$
\end{mybox}
\end{enumerate}
    %\newpage
    \item (20 points) Suppose $Y_1,\; Y_2,\; Y_3\;$ and $Y_4$ is an iid random sample from an exponential distribution with unknown rate parameter $\beta > 0$:
$$f(y) = \dfrac{1}{\beta}e^{-y/\beta}, \quad 0 < y < \infty .$$
Consider the two estimators of $\beta$: $\that_1 = \ybar$ and $\that_2 = \dfrac{2Y_1 + 3Y_2}{5}$.
\begin{enumerate}[label=(\alph*)]
    \item Show that $\that_2$ is an unbiased estimator of $\beta$.
    \begin{mybox}
        In order to show unbiased, we need $\operatorname{B}(\that_2) = \Es{\that_2} - \Es{\that} = 0$. Exponential is a gamma with $\alpha = 1$. Hence $\mu = \alpha \beta = \beta$ and $\sigma^2 = \alpha \beta^2 = \beta^2$. Computing the expected value, 
        $$
            \Es{\that_2} \;=\; \Eb{\dfrac{2Y_1 + 3Y_2}{5}} \;=\; \frac{2}{5}\Eb{Y_1} + \frac35 \Eb{Y_2} \;=\; \frac{2}{5}\beta + \frac35 \beta \;=\; \beta$$
        Thus $\operatorname{B}(\that_2) = \Es{\that_2} - \Es{\that} = \beta - \beta = 0$. Hence, $\that_2$ is unbiased.
    \end{mybox}
\vspace{1.2in}
    \item Determine the efficiency of $\that_2$ relative to $\that_1$.
    \begin{mybox}
        The definition of efficiency is $\eff{\that_2}{\that_1} = \dfrac{\Vs{\that_1}}{\Vs{\that_2}}$, so we will compute the respective variances.
        \begin{align*}
            \Vs{\that_1} &= \Vs{\ybar} = \Vb{\over 4 \sum_{i=1}^4 Y_i} = \over{4^2} \Vb{Y_1 + Y_2 + Y_3 + Y_4} \\ &= \frac{\Vb{Y_1} + \Vb{Y_2} + \Vb{Y_3} + \Vb{Y_4}}{16} = \frac{4\beta^2}{16} = \frac{\beta^2}{4}
        \end{align*}
        \begin{align*}
            \Vs{\that_2} &= \Vb{\dfrac{2Y_1 + 3Y_2}{5}} = \Vb{\frac{2}{5}Y_1} + \Vb{\frac{3}{5}Y_2} \;=\; \frac{4}{25} \Vb{Y_1} + \frac{9}{25}\Vb{Y_2}\\
            &=  \frac{4}{25} \beta^2 + \frac{9}{25}\beta^2 \;=\; \frac{13}{25}\beta^2
        \end{align*}
        \begin{align*}
            \eff{\that_2}{\that_1} &= \dfrac{\Vs{\that_1}}{\Vs{\that_2}} = \dfrac{\beta^2/4}{13\beta^2/25} = \over4 \times \frac{25}{13} = \frac{25}{52}
        \end{align*}
    \end{mybox}

    \item Now consider a third estimator of $\beta$, $\that_3 = \min \pars{Y_1, \; Y_2, \; Y_3, \; Y_4}$. Show that the distribution of $\that_3$ is also exponentially distributed. (Recall the general density function for min order statistic is $f_{(1)}(y) = n \brac{1-F(y)}^{n-1}f(y)$.)
    \begin{mybox}
        From gamma with $\alpha = 1$ we have $f(y) := \over{\beta}e^{-y/\beta},\;y\in(0,\infty)$. Computing the antiderivitive, 
    \begin{align*}
        F(y) &= \int_0^y \frac{1}{\beta}e^{-y/\beta}\,dy \hspace{1.1in} g(y) = u = -\frac{y}{\beta}\\
        &= \over{\beta} \int_{g(0)}^{g(y)} -\beta e^u du \hspace{1in} du = -\over{\beta}dy \iff dy = -\beta du\\
        &= - \int_0^{-y/\beta} e^u du = -\bbrac{e^u}_{u=0}^{u=-\frac{y}{\beta}} = - \bbrac{e^{-\frac{y}{\beta}} - 1} = -e^{-\frac{y}{\beta}} + 1.
    \end{align*}
    Substituting into the formula, 
    $$f_1(y) = 4\bbrac{1- \pars{-e^{-y/\beta} + 1}}^{3} \frac{e^{-y/\beta}}{\beta} = 4\pars{e^{-y/\beta}}^3 \frac{e^{-y/\beta}}{\beta} = \frac{4}{\beta}e^{-4y/\beta}.$$
    Hence it is still an exponential distribution with $\lambda = \dfrac{4}{\beta}$.
\end{mybox}
\vspace{.5in}
    \item Show that $\that_3$ is a biased estimator and compute the mean square error of $\that_3$.
\begin{mybox}
    $$\Eb{f_{(1)}(y)} = \Eb{ \frac{4}{\beta}e^{-4y/\beta}} = \Eb{\frac{1}{\beta/4}e^{-y/(\beta/4)}} = \frac{\beta}{4}$$
    $$\operatorname{B}(\that_3) = \Es{\that_3} - \Es{\that} = \frac{\beta}{4} - \beta = -\frac{3}{4}\beta \neq 0 \quad \therefore \quad \that_3 \text{ is biased}$$
    $$\Vs{\that_3} = \pfrac{\beta}{4}^2 = \frac{\beta^2}{16}$$
    \begin{align*}
        \operatorname{MSE}[\,\that_3\,] &= \Vs{\that_3}+ \operatorname{B}[\,\that_3\,]^2\\
        &= \frac{\beta^2}{16} + \pars{-\frac{3}{4}\beta}^2 \\ 
        &= \frac{1}{16}\beta^2 + \frac{9}{16} \beta^2\\
        &= \frac{5}{8}\beta^2
    \end{align*}
\end{mybox}
\end{enumerate} 
    \newpage
    \item (25 points) Suppose that $\thru{X}$ is an iid sample from a Rayleigh distribution with parameter $\theta > 0$ unknown:
$$f(x) = \dfrac{2x}{\theta}e^{-x^2/\theta},\quad 0 < x < \infty.$$
Note that $\E{X} = \dfrac{\sqrt{\pi \theta}}{2},\; \E{X^2} = \theta, \; \E{X^3} = \dfrac{3\sqrt{\pi \theta^3}}{4},\; \text{and } \E{X^4} = \dfrac{\theta^4}{2}$. (You do not need to prove these facts.)
\begin{enumerate}[label=(\alph*)]
    \item Find the method of moments estimator $\theta_{\text{MOM}}$ for $\theta$.
    \begin{mybox}
        $\mu_1'= \Eb{X} = \dfrac{\sqrt{\pi \theta}}{2}$ and $m_1' = \displaystyle \over{n}\sum_{i=1}^n X_i = \xbar$. Equating them,
            $$\dfrac{\sqrt{\pi \theta}}{2} = \xbar 
            \;\iff\; \sqrt{\pi \theta} = 2\xbar 
            \;\iff\;  \pi \theta = 4\xbar^2
            \;\iff\;  \theta_{\text{MOM}} = \frac{4 \xbar^2}{\pi}$$
    \end{mybox}
    \vspace{.2in}
    \item Find and simplify the likelihood function $L(\thru{x} \mid \theta)$, complete the factorization, and determine a sufficient statistic for $\theta$.
    \begin{mybox}
        \begin{align*}
            L(\vec{x} \mid \theta) &= \prod_{i=1}^n f(x_i \mid \theta ) = \prod_{i=1}^n \dfrac{2x_i}{\theta}e^{-x_i^2/\theta} = \pfrac{2}{\theta}^n \prod_{i=1}^n x_i e^{-x_i^2/\theta} \\
            &= \pfrac{2}{\theta}^n e^{-(\sum_{i=1}^n x_i^2 )/\theta} \prod_{i=1}^n x_i = \pfrac{2}{\theta}^n e^{-S/\theta} \prod_{i=1}^n x_i \hspace{.5in} S:=\sum_1^n x_i^2
        \end{align*} 
        $$\therefore \quad g(S \mid \theta) = \pfrac{2}{\theta}^n e^{-S/\theta} \quad \text{and} \quad h(\vec{x}) = \prod_{i=1}^n x_i$$
        And $S$ is sufficient for $\theta$ by the Factorization Theorem.
    \end{mybox}
\vspace{.2in}
    \item Find the maximum likelihood estimator $\theta_{\text{MLE}}$ for $\theta$.
    \begin{mybox}
        \begin{align*}
            \ln(L(\theta)) &= n \ln \pfrac{2}{\theta} -\frac{1}{\theta}\sum_{i=1}^n x_i^2 + \ln \pars{\prod_{i=1}^n x_i} \\
            &= \underbrace{n \ln 2 - n \ln \theta}_{\text{log division law}} -\frac{1}{\theta}\sum_{i=1}^n x_i^2 + \sum_{i=1}^n \ln x_i\\
            \frac{\partial}{\partial \theta}\pars{\ln\big(L(\theta)\big)} &= 0 - \frac{n}{\theta} + \over{\theta^2} \sum_{i=1}^n x_i^2 + 0 = 0
            \iff \frac{n}{\theta} = \over{\theta^2} \sum_{i=1}^n x_i^2 \\
            \iff n\theta &=  \sum_{i=1}^n x_i^2
            \iff \theta_{\text{MLE}} = \over{n}\sum_{i=1}^n x_i^2
        \end{align*}


    \end{mybox}
    \item Show that the maximum likelihood estimator of $\theta$ is consistent.
    \begin{mybox}
        For a consistent estimator, we need $\displaystyle \limn \Vs{\theta_{MLE}} = 0$. Computing the variance,
        \begin{align*}
            \Vb{\theta_{\text{MLE}}} = \Vb{\over{n}\sum_{i=1}^n x_i^2} &= \over{n^2}\Vb{\sum_{i=1}^n x_i^2} = \over{n^2}\sum_{i=1}^n \Vb{x_i^2} = \over{n} \Vb{x_i^2}\\
            &= \over{n} \pars{\Eb{x_i^{2\cdot 2}} - \Eb{x_i^2}^2} = \over{n} \pars{\frac{\theta^4}{2} - \theta^2}\\
            \limn \Vs{\theta_{\text{MLE}}} &= \limn \over{n}\pars{\dfrac{\theta^4}{2} - \theta^2} = \pars{\dfrac{\theta^4}{2} - \theta^2} \limn \over{n} = 0.
        \end{align*}
        Therefore $\theta_{\text{MLE}}$ is consistent.
    \end{mybox}
\vspace{.5in}
    \item Is the maximum likelihood estimator a minimum variance unbiased estimator? Briefly explain your answer.
    \begin{mybox}
        $$\Eb{\theta_{\text{MLE}}} = \Eb{\over{n}\sum_{i=1}^n x_i^2} = \Eb{x_i^2} = \theta.$$
        $$\operatorname{B}[\,\theta_{\text{MLE}}\,] = \Eb{\theta_{\text{MLE}}} - \theta = \theta - \theta = 0 \quad \therefore \quad \text{unbiased.}$$
        Because $\theta_{\text{MLE}}$ is unbiased and $S$ is a sufficient statistic (part b), then our $\theta_{\text{MLE}}$ is an MVUE via the Rao-Blackwell Theorem. 
    \end{mybox}

    \end{enumerate} 
    \end{enumerate}
\end{document} 