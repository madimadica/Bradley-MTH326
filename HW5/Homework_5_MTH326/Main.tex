\documentclass[12pt]{article}
\usepackage[english]{babel}
\usepackage{array}
\usepackage{setspace}
\usepackage{graphicx}
\usepackage{sistyle} %\num{100000} for commas
\SIthousandsep{,}
\usepackage{fancyhdr}
\usepackage{listings} % For code listings, may break stuff
\usepackage{xcolor, diagbox, empheq, makecell, tcolorbox}
\usepackage[autostyle]{csquotes}
\usepackage{amssymb, amsthm, linguex, enumitem, amsmath}
\usepackage{tcolorbox} %dont know what this does
\usepackage[colorlinks=true, allcolors=blue]{hyperref}
\usepackage{lipsum}
\usepackage{physics}

\makeatletter   %% <- make @ usable in macro names
\newcommand*\notab[1]{%
  \begingroup   %% <- limit scope of the following changes
    \par        %% <- start a new paragraph
    \@totalleftmargin=0pt \linewidth=\columnwidth
    %% ^^ let other commands know that the margins have been reset
    \parshape 0
    %% ^^ reset the margins
    #1\par      %% <- insert #1 and end this paragraph
  \endgroup
}
\makeatother    %% <- revert @

\pagestyle{empty}

\textwidth 6.5in
\hoffset=-.65in
\textheight=9.5in
\voffset=-1.in

%Sets
\newcommand{\R}{\mathbb{R}}
\newcommand{\C}{\mathbb{C}}
\newcommand{\N}{\mathbb{N}}
\newcommand{\F}{\mathbb{F}}
\newcommand{\Cb}{\mathbf{C}}
\newcommand{\Fb}{\mathbf{F}}
\newcommand{\Rb}{\mathbf{R}}

%Misc
\newcommand{\pars}[1]{\left( {#1} \right) } %auto size parenthesis 
\newcommand{\brac}[1]{\left[ {#1} \right] } %auto size brackets around arg
\newcommand{\set}[1]{\left\{{#1}\right\}} %auto size curly braces around arg
\newcommand{\vbrac}[1]{\left\langle{#1}\right\rangle} %vector angle brackets

\newcommand{\ceiling}[1]{\left\lceil {#1} \right\rceil} %auto size ceiling around arg
\newcommand{\floor}[1]{\left\lfloor {#1} \right\rfloor} %auto size floor around arg

\newcommand{\limn}{\lim_{n\to\infty}} %limit as n approaches infinity
\newcommand{\limx}{\lim_{x\to\infty}} %limit as n approaches infinity
\newcommand{\thru}[1]{{#1}_1, \dots, {#1}_n}
\newcommand{\sumthru}[1]{{#1}_1 + \cdots + {#1}_n}
\renewcommand{\over}[1]{\frac{1}{{#1}}}
\newcommand{\pfrac}[2]{\left( \frac{{#1}}{{#2}} \right) } %auto size parenthesis over fraction 
\newcommand{\pover}[1]{\left( \frac{1}{{#1}} \right) } %auto size parenthesis over fraction

%Boolean Algebra
\newcommand{\OR}{\,\lor\,}
\newcommand{\AND}{\,\land\,}

%Probability and Statistics
\newcommand{\xbar}{\bar{X}}
\newcommand{\ybar}{\bar{Y}}
\newcommand{\yn}{Y_1, \dots, Y_n}
\newcommand{\yx}{X_1, \dots, X_n}
\newcommand{\normDistT}{N\left(\mu, \sigma^2\right)} %default normal distribution
\newcommand{\gammaDist}[2]{\operatorname{Gamma} \left( {#1},{#2} \right)}
\newcommand{\normDist}[1]{N \left( {#1} \right)}
\newcommand{\prob}[1]{P \left( {#1} \right) }
\newcommand{\E}[1]{E \left( {#1} \right) }
\newcommand{\Eb}[1]{E[ \,{#1}\, ]} %E bracket
\newcommand{\Bb}[1]{B[ \,{#1}\, ]} %E bracket
\newcommand*{\V}[1]{V \left( {#1} \right) }
\newcommand{\Vb}[1]{V [ \,{#1}\, ] }
\newcommand{\that}{\hat{\theta}} %theta hat
\newcommand{\phat}{\hat{p}}
\newcommand{\psihat}{\hat{\psi}}
\newcommand{\Psihat}{\hat{\Psi}}

%Linear Algebra
\newcommand{\poly}[1]{\mathcal{P}_{#1}(\mathbf{R})} %polynomial up to degree (arg)
\newcommand{\pf}{\mathcal{P}(\mathbf{F})} %set of all polynomials
\newcommand{\Lc}{\mathcal{L}} %Set of linear maps
\newcommand{\vdub}[2]{\begin{pmatrix}{#1}\\{#2}\end{pmatrix}} %vertical vector parenthesis, 3 args
\newcommand{\vtrip}[3]{\begin{pmatrix}{#1}\\{#2}\\{#3}\end{pmatrix}} %vertical vector parenthesis, 3 args
\newcommand{\vquad}[4]{\begin{pmatrix}{#1}\\{#2}\\{#3}\\{#4}\end{pmatrix}} %vertical vector parenthesis, 4 args
\newcommand{\vquin}[5]{\begin{pmatrix}{#1}\\{#2}\\{#3}\\{#4}\\{#5}\end{pmatrix}} %vertical vector parenthesis, 5 args
\newcommand{\vvector}[3]{\begin{bmatrix}{#1}\\{#2}\\{#3}\end{bmatrix}} %vertical vector braces, 3 args
\newcommand{\dimn}[1]{\operatorname{dim}\,{#1}}
\renewcommand{\rank}[1]{\operatorname{dim}\operatorname{range}{#1}}
\newcommand{\nullity}[1]{\operatorname{dim}\operatorname{null}{#1}}
\newcommand{\matx}[4]{\begin{bmatrix}{#1} & {#2}\\{#3}&{#4}\end{bmatrix}} % 2x2 matrix
\newcommand{\detx}[4]{\begin{vmatrix}{#1} & {#2}\\{#3}&{#4}\end{vmatrix}} % 2x2 determinant
\newcommand{\nf}{\infty}
%colors
\definecolor{ggreen}{RGB}{0, 127, 0}
\definecolor{dgray}{RGB}{63,63,63}
\definecolor{neonorange}{RGB}{255,47,0}
\definecolor{mygray}{rgb}{0.5,0.5,0.5}
\newcommand{\red}[1]{\color{red}{#1}\color{black}}
\newcommand{\grn}[1]{\color{ggreen}{#1}\color{black}}
\newcommand{\blu}[1]{\color{blue}{#1}\color{black}}
\newcommand{\redx}[1]{\color{red}\not{#1}\color{black}}

\newcommand{\say}[1]{\textquotedblleft{#1}\textquotedblright} %quote the "argument"
\newcommand*\widefbox[1]{\fbox{\hspace{2em}#1\hspace{2em}}}
\newtcolorbox{mybox}[1][]{colback=white, sharp corners, #1}

%Line break spacings
\newcommand{\nl}{\vspace{0.1in}\noindent}
\newcommand{\nnl}{\vspace{0.2in}\noindent}
\newcommand{\nnnl}{\vspace{0.3in}\noindent}

% Code snippets
\newcommand*{\code}{\fontfamily{qcr}\selectfont}
\lstset{
    backgroundcolor=\color{white},
    basicstyle=\footnotesize,
    breakatwhitespace=false,         % sets if automatic breaks should only happen at whitespace
    breaklines=true,                 % sets automatic line breaking
    captionpos=b,                    % sets the caption-position to bottom
    commentstyle=\color{dgray},    % comment style
    deletekeywords={...},            % if you want to delete keywords from the given language
    escapeinside={(*@}{@*)},          % if you want to add LaTeX within your code
    extendedchars=true,              % lets you use non-ASCII characters; for 8-bits encodings only, does not work with UTF-8
    firstnumber=1,                % start line enumeration with line 1
    frame=single,	                   % adds a frame around the code
    keepspaces=true,                 % keeps spaces in text, useful for keeping indentation of code (possibly needs columns=flexible)
    keywordstyle=\color{neonorange},       % keyword style
    language=C++,                 % the language of the code
    morekeywords={*,...},            % if you want to add more keywords to the set
    numbers=left,                    % where to put the line-numbers; possible values are (none, left, right)
    numbersep=5pt,                   % how far the line-numbers are from the code
    numberstyle=\tiny\color{mygray}, % the style that is used for the line-numbers
    rulecolor=\color{black},         % if not set, the frame-color may be changed on line-breaks within not-black text (e.g. comments (green here))
    showspaces=false,                % show spaces everywhere adding particular underscores; it overrides 'showstringspaces'
    showstringspaces=false,          % underline spaces within strings only
    showtabs=false,                  % show tabs within strings adding particular underscores
    stringstyle=\color{purple},     % string literal style
    tabsize=4,	                   % sets default tabsize to 4 spaces
}

\lstdefinestyle{cpp}{language=C++,
    morekeywords={cout, cin, Comparable, T},numbers=none
}
%Examples:
%{\code while}
%
%{\code \begin{lstlisting}[language=C++]
%sum1 = 0;
%for (i = 1; i <= n; i *= 2)
%    for (j = 1; j <= n; j++)
%        sum1++;
%\end{lstlisting}}







\begin{document}

\pagestyle{fancy}
\fancyhf{}
\fancyhead[RO]{Matthew Wilder} %header top right
\fancyhead[LO]{MTH 326 - Homework \#5} %header top left
\fancyfoot[CO]{Page \thepage} %page center bottom

\noindent MTH 326 - Spring 2022
\\Assignment \#5
\\Due: Friday, February 25, 2022 (23:59)\\

\begin{enumerate}
    \item Suppose $X_1, X_2, \dots, X_n$ are iid with the common density function
    $$f(x) = \frac{2\theta^2}{x^3}, \quad x > \theta$$
    where $\theta$ is unknown.
    \begin{enumerate}
        \item Find the method of moments estimator of $\theta$.
        
        \nl \textbf{Solution: } As per the notes,
$$\underbrace{\mu_k' =E[\,X^k\,]}_{\text{\color{red}population}} \qquad \text{and} \qquad \underbrace{m_k' = \over{n}\sum X_i^k}_{\text{\color{red}data set}}$$
Computing $\mu_1'$, we have
\begin{align*}
    \mu_1' &= \Eb{X}\\
    &= \int_{\theta}^{\infty} x \cdot \frac{2\theta^2}{x^3} \,dx\\
    &= 2\theta^2 \int_{\theta}^{\infty} x^{-2}\,dx\\
    &= 2\theta^2 \brac{-\frac{1}{x}}_{x=\theta}^{x=\infty}\\
    &= 2\theta^2 \brac{- \limx \over{x} + \frac{+1}{\theta}}\\
    &= \frac{0 + 2\theta^2}{\theta}\\
    &= 2\theta
\end{align*}
Then for $m_k'$,
\begin{align*}
    m_k' &= \over{n}\sum X_i^k = 2\theta\\
    &\iff m_k' = \xbar = 2\theta\\
    &\iff \boxed{\that = \frac{\xbar}{2}}
\end{align*}

        \newpage\item Find the bias and variance of your estimator.
        
        \nl \textbf{Solution: } By the definition for Bias, $B = \Eb{\that} - \theta$. Substituting in our estimator from part (a),
\begin{align*}
    \Bb{\that} &= \Eb{\that} - \theta & \text{ Definition of Bias}\\
    &= \E{\frac{\xbar}{2}} - \theta & \text{Substituting from (a)}\\
    &= \over{2} \Eb{\xbar} - \theta & \text{Linearity of E}\\
    &= \over2 \brac{\E{\over{n} \sum_{i = 1}^n X_i }} - \theta & \text{Definition of } \xbar \\
    &= \over{2n} \sum_{i = 1}^n \Eb{X_i} - \theta & \text{Linearity of } E\\
    &= \over{2n} \sum_{i = 1}^n 2\theta - \theta & \Eb{X} \text{ from (a)}\\
    &= \frac{n\cdot 2\theta}{2n} - \theta & \text{Sum } n \text{ times}\\
    &= \theta - \theta & \text{Cancel } 2n\\
    &= 0
\end{align*}
\begin{center} \fbox{Therefore $\that$ is an unbiased estimator.}
\end{center}
As for variance, 
$$\Vb{X_i} = \blu{\Eb{X_i^2}} - \grn{\Eb{X_i}^2}$$
\begin{align*}
    \blu{\Eb{X^2}} &= \int_{\theta}^{\infty} x^2 \cdot \frac{2\theta}{x^3}\,dx \\
    &= 2\theta^2 \int_{\theta}^{\infty} \over{x}\,dx\\
    &= 2\theta^2 \brac{\ln x}_{x = \theta}^{x = \infty}\\
    &= 2\theta^2 \brac{\limx \ln x - \ln \theta}\\
    &= \infty
\end{align*}
And $\grn{\Eb{X_i}^2} = \grn{(2\theta)^2} = \grn{4\theta^2}$. Therefore 
$$\Vb{X_i} = \blu{\infty} - \grn{4\theta^2} = \infty.$$
\begin{center} \fbox{Therefore the variance is infinite.}
\end{center}
    \end{enumerate}

    \newpage
    \item Suppose that $X_1, X_2, \dots, X_n$ are a random sample from an exponentially distributed population with unknown mean $\theta$. Find the maximum likelihood estimator of the population variance $\theta^2$. 

\nnl \textbf{Solution: } An exponential distribution is a Gamma distribution with $\alpha = 1$, therefore
$$f(x) = \lambda e^{-\lambda x} \qquad x \in [0, \infty) \qquad \mu_x = \over{\lambda}$$
If $\theta$ is the mean then $\theta = \over{\lambda}$. Solving for $\lambda$, $\lambda = \over{\theta}$. Substituting this into the exponential formula,
$$f(x) = \frac{e^{-x / \theta}}{\theta}.$$
Now we can start to find the likelihood estimator,
\begin{align*}
    L(\thru{X} \mid \theta) &= \prod_{i=1}^n f(X_i \mid \theta)\\
    &= \frac{e^{-x_1/\theta}}{\theta} \times \cdots \times \frac{e^{-x_n/\theta}}{\theta}\\
    &= \frac{e^{-x_1/\theta} \times \cdots \times e^{-x_n/\theta  }}{\theta^n}\\
    &= \frac{e^{\frac{-x_1 - x_2 - \cdots - x_n}{\theta}}}{\theta^n}\\
    &= \frac{e^{-\xbar\,/\,\theta}}{\theta^n}
\end{align*}
Then, taking the log likelihood function,
\begin{align*}
    \ln \pars{L(\thru{X} \mid \theta)} &= \ln \pars{ \frac{e^{-\xbar\,/\,\theta}}{\theta}}\\
    &= \ln \pover{\theta^n} + \ln \pars{e^{-\xbar\,/\,\theta}}\\
    &=  \ln \pars{\theta^{-n}} + \pfrac{-\xbar}{\theta}\ln \pars{e^1}\\
    &= -n \ln \theta + \pfrac{-\xbar}{\theta}\cdot (1)\\
    &= -n \ln \theta - \frac{\xbar}{\theta}.
\end{align*}
Differentiating said log likelihood function,
\begin{align*}
    \pdv{\ln L}{\theta} &= \pdv{}{\theta}\pars{-n\ln \theta} - \pdv{}{\theta}\pfrac{\xbar}{\theta}\\
    &= -\frac{n}{\theta} + \frac{\xbar}{\theta^2}
\end{align*}
To maximize (or minimize) $\theta$, we let $-\frac{n}{\theta} + \frac{\xbar}{\theta^2} = 0$ and solve.
\begin{align*}
    -\frac{n}{\theta} + \frac{\xbar}{\theta^2} = 0 &\iff \frac{\xbar}{\theta^2} = \frac{n}{\theta} \\
    &\iff \xbar \theta = n\theta^2\\
    &\iff \xbar = n\theta\\
    &\iff \theta = \frac{\xbar}{n}
\end{align*}

\nl By the invariance property, the MLE of $\theta^2$ is the square of the MLE of $\theta$. So,
\begin{align*}
    & \operatorname{MLE}(\theta^2) = \pars{\operatorname{MLE}(\theta)}^2\\
    & \iff \boxed{\operatorname{MLE}(\theta^2) = \pars{ \frac{\xbar}{n}}^2}
\end{align*}
    
    \newpage
    \item Let $X_1, X_2, \dots, X_n$ denote a remote sample from the density function given by
    $$f(x) = \pover{\theta} rx^{r-1}e^{-x^r/\theta},\qquad x > 0, \qquad \theta > 0$$
    where $r$ is a known positive constant. Consider the statistic defined by $U = \dfrac{1}{n} \sum X_i^r$.
    \begin{enumerate}
        \item Find and simplify the likelihood function $L(x_1,x_2,...,x_n \mid \theta)$ and complete the factorization.
        
        \nl \begin{align*}
    L(\thru{X} \mid \theta) &= f(X_1 \mid \theta) \times \cdots \times f(X_n \mid \theta)\\
    &= \prod_{i=1}^{n} f(X_i \mid \theta)\\
    &= \prod_{i=1}^{n} \over{\theta} f(X_i)\\
    &= \over{\theta^n}\prod_{i=1}^{n} f(X_i)\\
    &= \over{\theta^n}\prod_{i=1}^{n} rx_i^{r-1}e^{-x_i^r/\theta}\\
    &= \frac{e^{-\xbar^r / \theta}r^n}{\theta^n}\prod_{i=1}^{n} x^{r-1}_i
\end{align*}
To factorize, we will let
$$g(U, \theta) = \over{\theta^n} e^{-\xbar^r/\theta} \qquad \text{and} \qquad h(\thru{X})  = r^n \prod_{i=1}^n x^{r-1}_i$$
\vspace{0.5in}
        \item Show that $U$ is a sufficient statistic for $\theta$.
         
        \nl Since $U = \dfrac{1}{n} \sum X_i^r$, then 
$$g(U, \theta) = \over{\theta^n} e^{-\xbar^r / \theta} \qquad \text{By factorization theorem.}$$

\nl Since $\xbar$ is in $g$, then $\xbar = \sum_{i=1}^n X_i^r$ is sufficient by the factorization theorem.

        \newpage  \item Show that $U$ is the MLE for $\theta$.
         
        \nl $$L(\thru{X} \mid \theta) = \frac{r^n \cdot e^{-\xbar^r/\theta}}{\theta^n} \prod_{i=1}^n x^{r-1}_i $$ 
\begin{align*}\ln \pars{L(\thru{X} \mid \theta)} &= \ln\pover{\theta}^n + \ln r^n + \ln \pars{ \prod_{i=1}^n x^{r-1}_i } + \ln \pars{ e^{-\xbar^r/\theta}}\\
    &= -n \ln \theta + n \ln r + (r-1)\ln \xbar - \frac{\xbar^r}{\theta} 
\end{align*}

\begin{align*}
    \pdv{}{\theta}\bigg[\ln \Big(L(\thru{X} \mid \theta)\Big)\bigg] &= -\frac{n}{\theta} + 0 + 0 + \frac{\xbar^r}{\theta^2}
    \\ &= 0
    \\ & \iff  \frac{\xbar^r}{\theta^2} = \frac{n}{\theta}\\
    & \iff \theta = \frac{\xbar^r}{n}
\end{align*}

        \vspace{1in}
        \item Show that $U$ is an unbiased estimator of $\theta$.
        
        \nl \begin{align*}
    \Eb{U} &= \E{\over{n}\sum_{i=1}^n X_i}\\
    &= \over{n} \sum_{i=1}^n \Eb{X_i}\\
    &= \over{n} \cdot n \theta\\
    &= \theta
\end{align*}
Therefore $U$ is unbiased.
\newpage
        \item What can we now conclude about the estimator $U$.
         
        \nl Because $U$ is unbiased, sufficient, and has the MLE; we can conlude that $U$ is the MVUE (minimum variance unbiased estimator).

        \vspace{.5in}
        \item Explain why $V(U)$ is finite. (You do not have to compute it, but clearly explain
        why it is finite.)
         
        \nl The variance of $U$ is finite because we showed $U$ to be an MVUE and all one-variable functions with an MVUE guarantee the existance of finite variance. 

\nl Analytically, suppose that $U$ had infinite variance, then $U$ couldn't have an MVUE since the minimum \textit{variance} of infinity would be infinite; and thus not the minimum.
        \vspace{.5in}
        \item Show that $U$ is a consistent estimator of $\theta$.
         
        \nl We first need the variance:
\begin{align*}
    \Vb{U} &= \V{\over{n} \sum_{i=1}^n X_i^r}\\
    &= \over{n^2} \sum_{i=1}^{n} \V{X_i^r}\\
\end{align*}
Taking the limit,
$$\limn \Vb{U} = \limn \pars{\over{n^2} \sum_{i=1}^n \V{X_i^r}}$$
We know that $\limn \dfrac{1}{n^2} = 0$. Since we showed in (f) that $\Vb{U}$ is finite, $\Vb{X}$ is also finite since $U$ is a sum of $X$'s. 0 times a finite quantity is going to be 0. Therefore $\Vb{U} = 0$. Since the variance is zero, $U$ is a consistent estimator of $\theta$.

    \end{enumerate}
\end{enumerate}
\end{document} 